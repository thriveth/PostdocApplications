\documentclass[12pt, a4paper]{scrartcl}
\usepackage{graphicx}
\usepackage{amssymb}
\usepackage{epstopdf}
\DeclareGraphicsRule{.tif}{png}{.png}{`convert #1 `dirname #1`/`basename #1 .tif`.png}

% \usepackage{titlesec}
% \titleformat{\subsection}[runin]{\sffamily \bfseries \large}{}{}{}[]
\renewcommand*{\titlepagestyle}{scrheadings}
\usepackage{aas_macros}
\usepackage{palatino}
\usepackage{multicol}
\usepackage[]{natbib}
\usepackage[usenames,dvipsnames,svgnames,html]{xcolor}
\definecolor{theblue}{rgb}{0.12156862745098039, 0.47058823529411764, 0.7058823529411765}
\usepackage{tikz}
\bibstyle{apj}
\usepackage{amssymb}
\usepackage[utf8x]{inputenc}

\usepackage{setspace}  % Needed for Double Spacing
\usepackage[english]{babel}
\usepackage{lipsum}

\DeclareGraphicsRule{.tif}{png}{.png}{`convert #1 `dirname #1`/`basename #1 .tif`.png}
\usepackage{hyperref}

%\title{Galaxy evolution through the lens}  % add title here
%\author{T. Emil Rivera-Thorsen}  % include Author's name here
\date{}                                           % Activate to display a given date or no date

%%% BEGIN HYPERREF
\hypersetup{
	colorlinks=true,
	linkcolor=blue,
	urlcolor=blue,
	citecolor=black,
}
%%% END HYPERREF

\title{Star-forming galaxies across cosmic eras}

%%% BEGIN KOMASCRIPT
\setkomafont{title}{\sffamily \bfseries \color[HTML]{1F78B4}}
\setkomafont{captionlabel}{\small \bfseries \sffamily}
\setkomafont{caption}{\footnotesize}
\setcapindent{1em}
\setkomafont{section}{\Large \color[HTML]{1F78B4}}
\setkomafont{subsection}{\large \itshape \color[HTML]{1F78B4}}
\setkomafont{subsubsection}{\normalsize }% \color[HTML]{1F78B4}}
% \setkomafont{subsubsection}{\normalsize \itshape \color[HTML]{1F78B4}}
% \setkomafont{author}{\small \itshape}
\setkomafont{pagenumber}{\large \upshape \bfseries \color[HTML]{FFFFFF}}
\usepackage[automark]{scrpage2}
\pagestyle{scrheadings}
\setheadsepline{.0pt}
\clearscrheadings
\automark[section]{chapter}
\ihead{\upshape \color[HTML]{999999} T. Emil Rivera-Thorsen}
\ohead{\colorbox[HTML]{1F78B4}{\color{white} \pagemark}}
%\ohead{\textcolor[HTML]{1F78B4}{\pagemark}}
%\ohead{
%    \begin{tikzpicture}
%	    \node[circle,draw,fill=theblue, color=theblue]{\color{white} \pagemark};
%    \end{tikzpicture}  
%    }
\chead{\upshape \color[HTML]{999999} Project description}
% \cfoot{}

%%% END KOMASCRIPT
\begin{document}

%\maketitle
% \noindent {\color[HTML]{1F78B4}\sffamily \bfseries \Large 
% Research and interests}

\section{Purpose and aims}

We propose a comparative study of rest-frame ultraviolet (UV) and optical
spectra of a sample of strongly star-forming galaxies in the local Universe and
a sample of gravitationally lensed galaxies at redshifts $1 \lesssim z \lesssim
3$, corresponding to a time span from around 15\% to 40\% the age of the
Universe, the time at which the majority of stars in the Universe were formed. 

For the local Universe, we wish to combine archival far-UV spectra from the
Hubble Space Telescope (HST) Cosmic Origins Spectrograph (COS), as well as an
on-going observational campaign with same instrument, and ancillary data from
the Sloan Digital Sky Survey (SDSS). 

At high redshifts, we have a sample of gravitationally lensed galaxies Observed
with spectrographs at the Keck and Magellan observatories. Due to the
gravitational lensing effect, these objects are spatially resolved and have very
high signal-to-noise ratios, allowing to study them in a level of detail usually
not attainable at these distances. We propose to give a standardized treatment
of galaxies at all redshifts, allowing for direct comparative studies of their
properties and enabling us to identify and characterize evolutionary trends.
Furthermore, we aim to employ the results of these comparative studies to
improve the understanding of spectral features which are expected to become
extremely important tracers of cosmological parameters  in the earliest epochs
of the Universe with the launch of the future James Webb Space Telescope, 
scheduled in 2018. 

The project is expected to \textbf{a)} improve our understanding of star
formation in the Universe and the highly
complex processes governing gas in star-forming galaxies,
\textbf{b)} study how these have evolved over cosmological time spans, and
\textbf{c)} use these insights to help design observation strategies
for the first galaxies in the Universe with the future James Webb Space
Telescope (JWST), scheduled for launch in 2018.


\section{Survey of the field} 

Some galaxies undergo episodes of extremely high-paced
star formation, during which their neutral gas is turned into stars at
rates much higher than in typical, quiescent galaxies. The high star formation
rates mean that these galaxies also have an extraordinary large population of
the hottest and most massive, but also very short lived, type O and B stars.
These stars  radiate a large part of their energy output in the hard ultraviolet
wavelength ranges. This highly energetic radiation strongly affects the
surrounding gas, ionizing and heating it, resulting in strong line emission by 
cascading recombining or decaying, collisionally excited electrons. Radiative
pressure and stellar wind from the hot stars and kinetic energy from frequent
supernova blasts stir up the gas, resulting in bulk outflows and strong velocity
gradients in the neutral gas. The star formation episodes are often triggered by
merging or interactions with other galaxies of intergalactic gas, which further
adds to the kinematic complexity of these galaxies. The feedback from young
stars and supernovae can accelerate substantial bodies of gas to escape
velocity, 

The spectrum of starburst galaxies is dominated by the smooth continuum emission
of young, hot type OB stars, with strong emission peaks from atomic and ionic
lines superimposed onto it.  The intrinsically strongest of these lines by a
broad margin is Lyman $\alpha$, the transition between the ground state and the
first excited level of neutral Hydrogen. As much as 2/3 of all ionizing photons,
and 1/3 of the total ionizing energy, gets reprocessed into this single, narrow
emission feature \citep{DijkstraRev}. The strength means that this line is often
detectable in narrow filters when the galaxy is otherwise too faint to be
observed in stellar continuum or other emission lines, and its location in the
far-UV range means that it stays within the transparent windows of the Earth's
atmosphere even at high redshifts. This renders Ly $\alpha$ a crucially
important tool for detecting galaxies of low stellar mass but strong star
formation -- which are abundant in the early universe -- and for
spectroscopically verifying redshifts, which can only be determined very crudely
by imaging techniques.  In cosmology, this helps more accurately mapping galaxy
positions, masses, clustering properties, etc.  and thus trace not only galaxy
evolution, but also cosmological phenomena like structure formation and the
ionization history of the Universe.

Lyman $\alpha$ is however a  strongly resonant line, interacting strongly with
the neutral gas in the galaxies where it arises, and is very vulnerable to
absorption by dust. The strong scattering affects both the observed morphology,
strength and spectral line profile of the transition in ways depending
intricately on a multitude of parameters like neutral gas column density,
clumping, bulk outflows, dust content, kinematic line widths and others
\citep[e.g.][]{Wofford2013,Atek2008,Atek2009,RiveraThorsen2015,Kunth1998,
Giavalisco1996,LARSI, LARSII}, and the observed properties in Lyman $\alpha$ are
statistically almost completely decoupled from its intrinsic properties. For the
Lyman-$\alpha$ observed to be correctly interpreted -- e.g. regarding how large
a fraction of galaxies at high redshifts are expected to be detectable in a
given survey, or what the typical masses of said galaxies are, etc. -- it is
necessary to know which mechanisms regulate Lyman $\alpha$ radiative transfer
and escape, and how they interact. Multiple studies have been made to map this
\citep[e.g][]{Atek2008,Atek2009,Hayes2005,Hayes2007,Hayes2009,Ostlin2009,Kunth1998}.
The most ambitious such study is the Lyman Alpha Reference Sample
\citep[LARS, ][]{LARS0,LARSI,LARSII}, a study of a sample of intrinsically Ly
$\alpha$-bright, strongly star forming galaxies at multiple wavelengths and
using using multiple instruments, from X-ray observations to 21 cm radio
interferometry. The aim was to understand these galaxies in depth and create a
baseline of comparison to observations at high redshifts. For this project, I
led the production of \cite{RiveraThorsen2015}, in which we analyzed far-UV
absorption lines in spectra of the sample galaxies obtained with HST-COS. We
analyzed the connections between physical parameters like line width, outflow
velocity, column density, and with global parameters determined from imaging
\citep{LARSII} and radio observations \citep{LARSIII}, and basic properties of
the observed Lyman $\alpha$ line. 

The galaxies of LARS are however only a fraction of the low redshift
star-forming galaxies that have been observed with HST-COS. A number of other
samples, including the references mentioned above, exist, all selected from
various criteria but with a moderate-to-strong star formation activity,
totalling around 80 galaxies by out current estimate. A uniform and carefully
designed analysis of these in the same way we did for \cite{RiveraThorsen2015}
and \cite{RiveraThorsen2017} is the aim of the first partial project of this
proposal. We also plan to complement this research with a single-object study of
unprecedented depth and detail, through data currently being acquired with
Hubble Space Telescope. What the archival study provides in width, this study
provides in depth; and together, they could yield very important insights into
the nature of starburst galaxies and the connections between star formation and
the gas from which they form.

Galaxies at high redshifts ($z \gtrsim 1$) are generally too distant to be
resolved even with the best telescopes, or just barely resolvable. However, if
the line of sight to the galaxies have a massive galaxy or galaxy cluster, the
gravity of this can bend the light from the background source and act as a
naturally occurring lens, magnifying the image and amplifying its light by
orders of magnitude. \textbf{Project Megasaura} is a sample (PI: Dr. Jane Rigby,
NASA Goddard Space Flight Center) of 17 such gravitationally lensed galaxies at
redshifts $1 \lesssim z \lessim 3$. Thanks to the strong lensing, these galaxies
are sufficiently bright that they can be observed in strong continuum, such that
an absorption line analysis is possible, which does not happen often at these
distances. Furthermore, the data quality is good enough that spatially resolved
studies are possible. The spectra cover a rest-frame wavelength range
containing, but not limited to, the one of most of the local-universe HST-COS
spectra, containing most of the spectral features of the local galaxies,
allowing for an apples-to-apples comparison of spectral features between the
high and low redshift samples. In particular, the SNR is good enough to
determine systemic zero-point velocities for the galaxies from stellar
absorption and nebular emission, something which has been a problem for past
works of this kind \citep[e.g.][]{Jones2013}. Furthermore, the Lyman-$\alpha$
lines are in some cases well enough resolved that it can be compared to a grid
of existing semi-analytical outflow and radiative transfer models
\citep{Schaerer2011}. This is to our knowledge the first time it is possible to
run analyses designed for the data quality of local-universe objects to be put
to use on objects at such large distances, and it would be of great interest to
see how the connections between ISM conditions and Lyman $\alpha$ line
properties compare to those we know from the local Universe.

The spectra for Project Megasaura also contain a number of near-UV emission
lines which are not usually observed in local samples due to the wavelength
coverage of the detector at COS. Most important is the semi-forbidden line
C\textsc{iii}] $\lambda$ 1909. As surveys push past redshifts $z \gtrsim 7$, we
look into a time when the first galaxies had not yet fully ionized the
intergalactic gas in the Universe, and the amounts of neutral gas present in the
Universe at these early times strongly suppress Ly$\alpha$ which we rely on at
lower redshifts. Spectroscopic confirmation of redshifts this high must rely on
detection of alternative rest-frame UV emission lines, and the C\textsc{iii}]
1909 and its forbidden [C\textsc{iii}] 1907 line are promising candidates
\citep{Stark2014,Jaskot2016}. The lines are however not universally present in
star-forming galaxies \citep{Rigby2015}; the line strength seems to be
correlated with extreme starburst conditions like high ionization, low
metallicity, and hard UV spectrum \citep{Stark2014,Jaskot2016}. The presence of
these lines in combination with the ones used for our usual diagnostics will
allow to establish better connections between these lines and physical
conditions in the ISM of the galaxies, and establish or describe with better
precision which biases etc. are related to the detection of these lines: Which
galaxies are detected in these lines, how strongly, which galaxies and how many
go undetected, etc. These insights are of great importance for designing
observation strategies for galaxies at extremely high redshifts with the
upcoming James Webb Space Telescope, and this research could help put us at the
forefront of research with this extremely important telescope once it is
launched in 2018.


\section{Project description}

We wish to carry out a project with two main parts: one at low redshifts and one
at high redshifts. Both aim to understand the intricate and complex mechanisms
that govern star formation and feedback, and some of their most important
observational signatures. The datasets available provide us with a unique
opportunity to analyze these galaxies in a uniform way and thus draw
evolutionary connections in these mechanisms from times when the universe was
only around a quarter of its current age, and to the present day. This is
possible thanks to the natural lensing effect of the gravity of foreground
galaxies, which enhance and amplify the light of these distant galaxies enough
to give a much higher data quality than could otherwise have been reached. 

\subsection*{Local universe}

\subsubsection*{Archival starburst galaxies with HST-COS}
Far-UV spectroscopic observations of star-forming galaxies in the local Universe
have been done for a number of years now, with various aims, purposes and
selection criteria\citep[e.g.][]{Heckman2011,Heckman2015,Alexandroff2015,
Wofford2013,Henry2015, RiveraThorsen2015}. For the first subproject, we wish to 
obtain archival data from a number of such campaigns including, but not limited
to, the ones mentioned above, totalling around 70 galaxies, all observed with
the Hubble Space Telescope Cosmic Origins Spectrograph, along with auxiliary
data from e.g. SDSS and other large samples where available. We will then analyze
these spectra in a uniform way, extending the coverage of galaxy types, masses
etc. of the Lyman Alpha Reference Sample, and dramatically improving the
statistical significance of the original sample. Besides measuring

\subsubsection*{SAFE: Star formation, lyman-Alpha, and Feedback in Eso-338}

Complementary to the statistical sample, this subproject takes the opposite
route by studying one single target in a detail unprecedented in the far
Ultraviolet wavelength range. 
SAFE is an ongoing campaign led by Prof. Östlin at Stockholm University, using
HST-COS to take far-UV spectra of a particularly interesting galaxy, known as
ESO 338-IG04, placing the circular aperture of the Cosmic Origins Spectrograph
on board the Hubble Space Telescope on 12 different locations of the galaxy,
covering both luminous regions of strong star formation, and the faint and
diffuse gas in the outskirts of the galaxy, allowing to perform a tomographic
study of its interstellar and circumgalactic gas. The method is akin to Integral
Field Spectroscopy, in which a spectrum is acquired for each pixel in an image
grid; but the far-UV wavelengths cannot penetrate the Earth's atmosphere, and no
spaceborne integral field units exist. Combined with results from previous
observations with e.g. HST imaging \citep{Hayes2009,Ostlin2009, Ostlin1998}, The
ESO X-Shooter optical spectrograph \cite[Rivera-Thorsen et al., submitted to
ApJ]{Guseva2012,Sandberg2013}, and the MUSE optical integral field spectrograph
\citep{Bik2015}, these observations are expected to yield important new insights
about stellar population, ISM distribution and dynamics, and Lyman $\alpha$
radiation transfer. 

\subsection*{High-redshift universe}

\subsubsection*{Project Megasaura} 

Project Megasaura is a sample of 17 star forming galaxies at redshifts between
1 and 3, corresponding to times when the Universe was between 15\% and 40\% of
its current age. These galaxies are strongly lensed by foreground galaxies or
clusters. The galaxies have been observed in the optical and infrared with
Magellan/MagE and Keck/ESI, and imaged by HST and Spitzer. The strong lensing
gives an unusually fine signal-to-noise, allowing for detailed studies in both
emission and absorption.  In itself, the sample is a unique opportunity to study
star-forming galaxies in the epoch where the majority of stars in the Universe
were formed, including stellar population, ionization conditions, electron
temperatures, detailed outflows, chemical enrichment, and more. However,
together with a local sample, it also holds the promise of disentangling
intrinsic, evolutionary changes over cosmic time from changes in cosmic
environment, and help understand star formation, galaxy evolution, and
Ly$\alpha$ transfer and escape both locally and in the early Universe.
Besides a comparative study of these properties at high and low redshifts, we
also plan to study in details some of the near-UV emission lines present in the
Megasaura datasets. As mentioned above, these lines are seen as promising tools
for observations at extremely high redshifts with JWST, and the presence of
these lines together with the ones used for our local-Universe diagnostics gives
us a unique opportunity gain insights about the physics governing the observable
fingerprints of these lines and thus help design observational strategies for
observing the first galaxies once JWST has been launched.

\subsection*{Preliminary timeline}

\textbf{\sffamily Months 1 -- 6} Local star-forming galaxies. This project is
fairly straightforward regarding methodology and well-defined in scope, and much
of the machinery for the analysis is already in place from earlier work
\citep{RiveraThorsen2015,RiveraThorsen2015}. However, some measure of manual
work is required, which will of course mean an increase in time requirements
with around 80 objects expected to be suitable for the project.\\
\textbf{\sffamily Months 7 -- 16} Megasaura I. Comparative study of lensed galaxies
		with the local galaxies studied in the first paper.This paper
		will present the sample and characteristics that are directly
		comparable to the low redshift sample like  characteristics like e.g.
		stellar population, gas and dust content, temperature, heavy
		element enrichment, characterization of gas in- or outflows
		etc., analogous but not limited to what is listed in Tables 2
		and 3 of \cite{LARSI} and parts of what is presented in
		\cite{RiveraThorsen2015}. Like the previous paper, we expect 
		there to be a comparatively large amount of work performing a
		large number of diagnostics and characterizations for the sample
		galaxies. \\
\textbf{\sffamily Months 17 -- 24} Megasaura II. In this project, we plan to focus 
		on the near-UV emission lines of the lensed galaxies. 
		Star-forming galaxies are not well studied in these lines; at 
		low redshifts they fall in a range of poor detector coverage of 
		the HST/COS; and at higher redshifts, they are often too weak 
		to be observed with current instruments. With the coming launch 
		of the JWST, however, these lines are going to play an important 
		part in pushing the limits for high-$z$ observations. 
		High-redshift galaxies are typically detected by their Lyman α 
		emission, but at redshifts beyond $z \sim 7$, the neutral fraction of
		the IGM is high enough to effectively quench the majority of 
		Ly$\alpha$ radiation.  The JWST will on the other hand be able 
		to reach much deeper and detect these metallic rest frame NUV 
		lines, like e.g. Mg II, C III], and Fe II. The combined findings 
		of Megasaura I and the Local HST Legacy Starbursts will provide 
		the foundation to better understand and interpret these lines.\\
\textbf{\sffamily Months 25 -- 30 (?)} SAFE I. SAFE contains enough data that two
		years' activity could probably be dedicated to analyzing this
		dataset alone. How much time should be dedicated to this depends
		on the initial results of data analysis, weighed up against
		which questions arise during the work on the two Project 
		Megasaura papers. A short version of a SAFE paper should at
		least consist of a mapping of the neutral and ionized ISM phases
		along the line of sight towards the clusters in the relevant
		pointings, a characterization of Lyman $\alpha$ in all
		COS pointings. This will provide a detailed mapping of
		properties directly comparable to, but more detailed than, the 
		ones known from the Lyman Alpha Reference Sample and from the 
		first projects of this proposal, and could help discern various
		possible explanations for e.g. observed Lyman $\alpha$. 
		A further elaboration on this work could either be in
		collaboration with modeling experts  \citep{Schaerer2011,
			Laursen2009, Laursen2009b, Gronke2016, Verhamme2006, Verhamme2008}
		or, if the data quality of the Megasaura galaxies permits it,
		comparative spatially-resolved studies based on these and SAFE,
		like an extended version of the work presented by
		\cite{Bordoloi2016}


\section{Significance}

This is to our knowledge the first time it will be possible to bring high
quality diagnostics designed for galaxies in the local Universe to use at such
high redshifts, which can yield important insights about the star formation
history  and galaxy evolution of the Universe.

The proposed project will also help to better understand the nature of star
forming galaxies in the local Universe by dramatically improving the sample size
and statistical robustness of existing studies like e.g.
\cite{RiveraThorsen2015}, and improve the coverage of galaxy types in terms of
e.g. stellar population, mass, ionization, metallicity, etc.

The SAFE sub-project will bring unprecedented detail in the study of the Lyman
$\alpha$ line and how it is influenced by gas properties in the emitting galaxy,
which in turn is important to our interpretation and understanding to large
amounts of cosmological data based in this line.

The direct comparison between observational signatures at high and low redshifts
will allow us to calibrate our understanding of near-UV emission lines which
will play a crucial role in future studies pushing the frontier of how far and
deep into the early Universe we can see with the launch of the James Webb Space
Telescope.


\section{Preliminary results}

The local starbursts subproject, which is planned to be the first step, builds
on analysis done and developed for the Lyman Alpha Reference Sample
\citep{RiveraThorsen2015}, but aims to extend the statistical robustness and
parameter space coverage of this work. This means that the methodology is tested
and known to work, and much of the machinery for this analysis is in place. 
This machinery will also be used for parts of the SAFE project and, in a shape
modified to be appropriate for the different data format etc., also for one part
of Project Megasaura. Much of the rest of the work will rely on well known
astrophysical diagnostics.


\section{Results}

With this project, we expect to gain important understanding of the conditions
under which the majority of stars in the Universe have formed. 

We expect to gain new insights about both local an distant galaxies
undergoing episodes of intense star formation, and how they influence and are
influenced by their environment, and to draw evolutionary connections from a
uniform analysis of galaxies at different redshifts.

We expect to get improved knowledge of connections between Lyman $\alpha$ escape
and a set of key observables in the local universe, how these change under
varying circumstances, and how they compare to the situations in the early,
high-redshift universe.

We expect to be able to answer at least a subset of these questions for galaxies
at larger distances and earlier phases in Cosmic history, and look at how these
connections and mechanisms have evolved in time. 

We expect to be able to draw connections between astrophysical properties of the
galaxies and the observational signatures of key emission features in the
near-UV, which are expected to be crucial for galaxy studies at extremely high
redshifts which will be made possible by the James Webb Space Telescope, and we
expect this to bring valuable contributions to the design of observation
strategies to look for the first galaxies in the Universe with JWST.


% \section*{Independent line of research}
% 
% This project is an individual project, conducted under the mentorship of
% senior researchers at Stockholm University and Goddard Space Flight Center. The
% data used is archival data (local starbusts project), data acquired by Dr.
% Rigby (project Megasaurs), or Prof. Östlin (SAFE), but the the analysis will be
% carried out and publications led by the applicant.

\newline
\vspace{0.5em}

\bibliographystyle{aasjournal}
\begin{scriptsize}
	\begin{multicols}{2}
	\bibliography{thesis}
	\end{multicols}
\end{scriptsize}

\end{document}  
